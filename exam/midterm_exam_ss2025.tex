% !TEX root = midterm_exam_ss2025.tex
% !TeX encoding = UTF-8
% !TeX spellcheck = en_US
\documentclass[a4paper]{scrartcl}
%%%%%%%%%%%%%%%%%%%%%%%%%%%%%%%%%%%%%%%%%%%%%%%%%%%%%%%%%%%%%%%%%%%%%%%%%%%%%%%%%%%%%%%%%%%%%%%%%%%%%%%%%%%%%%%%%%%%%%%%%%%%%%%%%%%%%%%%%%%%%%%%%%%%%%%%%%%%%%%%%%%%%%%%%%%%%%%%%%%%%%%%%%%%%%%%%%%%%%%%%%%%%%%%%%%%%%%%%%%%%%%%%%%%%%%%%%%%%%%%%%%%%%%%%%%%
\usepackage[T1]{fontenc}
%\usepackage[utf8]{inputenc}
\usepackage[bottom=2.5cm,top=2.0cm,left=2.0cm,right=2.0cm]{geometry}
\usepackage{amssymb,amsmath,amsfonts}
\usepackage[english]{babel}
\usepackage{enumitem}
\usepackage{booktabs}
\usepackage{csquotes}
\usepackage{url}
\usepackage{graphicx}
%\parindent0mm
%\parskip1.5ex plus0.5ex minus0.5ex
\usepackage[numbered,framed]{matlab-prettifier}

\begin{document}
	
\title{Computational Macroeconomics\\Midterm Exam}
\author{Willi Mutschler}
\date{Summer 2025\\~\\Version 1.0}
\maketitle

\section*{General information}

\begin{itemize}
\item
Answer \textbf{all} of the following \textbf{three} exercises in English.
\item
All assignments will be given the same weight in the final grade.
\item
Hand in your solutions before Friday June, 06 2025 at 3pm.
\item
The solution files should contain your executable (and commented) Dynare files, MATLAB functions and script files
  as well as all additional documentation as \texttt{pdf}, not \texttt{doc} or \texttt{docx}.
Your \texttt{pdf} files may also include scans or pictures of handwritten notes.
\item
Please e-mail ALL the solution files to \url{willi.mutschler@uni-tuebingen.de}.
\item
I will confirm the receipt of your work also by email (typically within the hour). If not, please resend it to me.
\item
All students must work on their own, please also give your student ID number.
\item
It is advised to regularly check Mattermost and your emails in case of urgent updates.
\item
If there are any questions, do not hesitate to contact Willi Mutschler.
\end{itemize}

\section*{Changelog}
\begin{itemize}
\item Version 1.0: First version of the exam.
\end{itemize}

\newpage

\section[An and Schorfheide (2007) Model]{An and Schorfheide (2007) Model\label{ex:AnScho}}
Consider the New Keynesian model of An and Schorfheide (2007, Econometric Reviews).
The model equations are given by 
\begin{align}
1 &= \beta \mathbb{E}_t\left[{\left(\frac{c_{t+1}}{c_t}\right)}^{-\tau} \frac{1}{\gamma z_{t+1}} \frac{R_t}{\pi_{t+1}}\right] \label{eq:AS_B1}
\\
1 &= \phi \left(\pi_t - \pi\right) \left[\left(1-\frac{1}{2\nu}\right)\pi_t + \frac{\pi}{2\nu}\right] - \phi \beta \mathbb{E}_t \left[{\left(\frac{c_{t+1}}{c_t}\right)}^{-\tau} \frac{y_{t+1}}{y_t} \left(\pi_{t+1} - \pi \right) \pi_{t+1}\right] + \frac{1}{\nu}\left[1-c_t^{\tau}\right]
\\
y_t &= c_t + \frac{g_t-1}{g_t} y_t + \frac{\phi}{2} {\left({\pi_t - \pi}\right)}^2 y_t
\\
R_t &= {R_t^{*}}^{1-\rho_R} R_{t-1}^{\rho_R} e^{\sigma_r\epsilon_{R,t}}
\\
\ln(z_t) &= \rho_z \ln(z_{t-1}) + \sigma_z\epsilon_{z,t}
\\
\ln(g_t) &= (1-\rho_g)\ln(\bar{g}) + \rho_g \ln(g_{t-1}) + \sigma_g\epsilon_{g,t}
\\
y_t^* &= {(1-\nu)}^{\frac{1}{\tau}} g_t
\end{align}
where \(\epsilon_{R,t}\), \(\epsilon_{g,t}\) and \(\epsilon_{z,t}\) are the exogenous variables.
Moreover, we have the following auxiliary parameter relationships:
\begin{align*}
\gamma = 1+\frac{\gamma^{Q}}{100}, \qquad
\beta = \frac{1}{1+r^{A}/400}, \qquad
\bar{\pi} = 1+\frac{\pi^{A}}{400}, \qquad
\phi = \tau \frac{1-\nu}{\nu \bar{\pi}^{2}\kappa}
\end{align*}
and one auxiliary variable \(R_t^*\):
\begin{align*}
R_t^* &= R {\left(\frac{\pi_t}{\bar{\pi}}\right)}^{\psi_1} {\left(\frac{y_t}{y_t^*}\right)}^{\psi_2}
\end{align*}
Note that \(y_t^*\) is an endogenous model variable, whereas \(\bar{\pi}\) and \(\bar{g}\) are parameters and \(R\) denotes the steady-state of \(R_t\).

\paragraph{Parametrization}
Use the following parameter values for the exercises:
\begin{align*}
&\tau=2,\quad
\kappa=0.33,\quad
\nu=0.1,\quad
\rho_g=0.95,\quad
\rho_z=0.9,\quad	
r^{A}=1,\quad
\\
&\pi^{A}=3.2,\quad
\gamma^{Q}=0.5,\quad
\bar{g}=1/0.85,\quad
\sigma_z = \sqrt{0.9},\quad
\sigma_g=\sqrt{0.36}
\\
&\psi_1=1.5,\quad \psi_2=0.125,\quad \rho_R = 0.75,\quad \sigma_r = \sqrt{0.4}
\end{align*}

\subsection*{Exercises}

\begin{enumerate}

\item
Provide intuition behind each equation of the model. How are nominal price rigidities modelled?\\\emph{Hint: Have a look at the relevant parts of the underlying paper.}

\item
Preprocess the model using the symbolic toolbox in MATLAB.

\item
Derive the steady-state analytically and write a function that computes it in {MATLAB}.

\item
Write a script that numerically finds the steady-state of the model in {MATLAB} using an optimizer of your choice.

\item
Calibrate such that the ratio of steady-state consumption to steady-state output is equal to 90\%.

\end{enumerate}

\newpage

\section[Monopolistic competition and irreversible investments]{Monopolistic competition and irreversible investments\label{ex:RBCModelMonCompIrrInvest}}

\begin{center} \Large \textbf{Model Description} \end{center}

\paragraph{Households: Utility}
The economy is assumed to be inhabited by a representative family with identical members.
The household's preferences are defined over consumption \(c_t\) and labor effort \(h_t\).
The representative household maximizes present as well as expected future utility
\begin{align}
\max E_t \sum_{s=0}^{\infty} \beta^{s} \left( \frac{c_{t+s}^{1-\sigma_c}}{1-\sigma_c} - \chi_h \frac{h_{t+s}^{1+\sigma_h}}{1+\sigma_h} \right)\label{eq:RBCMonopIrrInv.UtilityLifetime}
\end{align}
where \(E_t\) is the expectation operator conditional on information at time \(t\),
  \(\beta <1\) denotes the discount factor,
  \(\sigma_c\) and \(\sigma_h\) are elasticities and \(\chi_h\) the consumption weight.

\paragraph{Households: Capital Accumulation}
The household owns the (end of period) capital stock \(k_t\) which evolves according to
\begin{align}
k_t = (1-\delta)k_{t-1} + i_t \label{eq:RBCMonopIrrInv.CapitalAccumulation}
\end{align}
where \(\delta \) is the depreciation rate.

\paragraph{Households: Irreversible Investment}
An occasionally binding constraint is imposed that prevents investment \(i_t\) from falling below a fraction \(\omega \)
  of its value in the non-stochastic steady-state (denoted by \(i\)):
\begin{align}
i_t \geq \omega \cdot i \label{eq:RBCMonopIrrInv.IrrInvest}
\end{align}

\paragraph{Households: Budget}
Capital is rented to the intermediate firms at a nominal rate of \(R^k_{t}\) which the household takes as given when forming optimal plans.
Similarly, in each period the household takes the nominal wage \(W_t\) as given and supplies perfectly elastic labor service \(h_t\) to the firm sector.
In return they receive nominal labor income \(W_t h_t\).
All firms are owned by the household so that nominal profits and dividends from firms in the final good sector, \( Div^{Fin}_t\),
  as well as from each firm \(f\in[0,1]\) in the intermediate goods sector, \(\int_0^1 {Div}^{Int}_t(f)df\),
  are received by the household.
Income and wealth are used to finance consumption and investment expenditures, both are quoted at price level \(P_t\).
In total this defines the \emph{nominal} budget constraint of the household
\begin{align}
P_t c_t + P_t i_t =  W_t h_t + R^k_t k_{t-1} + Div^{Fin}_t + \int_0^1 Div^{Int}_t(f) df \label{eq:RBCMonopIrrInv.BudgetNominal}
\end{align}
In what follows, let lower case letters denote real variables, i.e.\
\begin{align*}
w_t = \frac{W_t}{P_t},\quad
r^k_t = \frac{R^k_t}{P_t},\quad
div^{Fin}_t = \frac{Div^{Fin}_t}{P_t},\quad
div^{Int}_t = \frac{Div^{Int}_t}{P_t}
\end{align*}

\paragraph{Final Good Firm (Retail Sector): Profit Maximization}
The economy is populated by a continuum of firms indexed by \(f \in [0,1]\) that produce differentiated goods \(y_t(f)\) in monopolistic competition.
The technology for transforming these intermediate goods into the final output good \(y_t\) that can be used for consumption and investment is given by:
\begin{align}
y_t = {\left[\int\limits_0^1 {(y_t(f))}^{\frac{\nu_t-1}{\nu_t}}df\right]}^{\frac{\nu_t}{\nu_t-1}} \label{eq:RBCMonopIrrInv.Firms.Aggregator}
\end{align}
where \(\nu_t\) is the time-varying elasticity of substitution between differentiated goods.

\paragraph{Intermediate Goods Firms (Wholesale Sector): Profit Maximization}
Intermediate firm \(f\in[0,1]\) uses the following production function to produce their differentiated good
\begin{align}
y_t(f) = a_t {(k_{t-1}(f))}^\alpha {(n_t(f))}^{1-\alpha} \label{eq:RBCMonopIrrInv.IntermediateFirms.ProductionFunction}
\end{align}
where \(a_t\) denotes the common technology level available to all firms.
Firms face perfectly competitive factor markets for renting capital \(k_{t-1}(f)\) and hiring labor \(n_t(f)\) with \(\alpha \) being a productivity parameter.
Nominal profits of firm \(f\) are equal to revenues from selling its differentiated good at self-determined price \(P_t(f)\)
  minus costs from hiring labor at real wage \(w_t\) and real renting rate of capital \(r^k_t\):
\begin{align}
{Div}^{Int}_t(f) = P_t(f) y_t(f) - P_t w_t n_t(f) - P_t r^k_t k_{t-1}(f) \label{eq:RBCMonopIrrInv.Firms.Profits}
\end{align}
The objective of the firm is to choose contingent plans for \(P_t(f)\), \(n_t(f)\) and \(k_{t-1}(f)\)
  so as to maximize the present discounted value of nominal dividend payments given by
\begin{align*}
E_t \sum_{s=0}^{\infty} \beta^s \frac{\lambda_{t+s+1}}{\lambda_{t+s}} Div^{Int}_{t+s}(f)
\end{align*}

\paragraph{Exogenous Variables}
The level of technology \(a_t\) and the elasticity \(\nu_t\) evolve according to
\begin{align}
\log{a_t} &= \rho_a \log{a_{t-1}} + \varepsilon_{a,t} \label{eq:RBCMonopIrrInv.LoM.TFP}
\\
\nu_t &= \bar{\nu} + \varepsilon_{\nu,t} \label{eq:RBCMonopIrrInv.LoM.Elast}
\end{align}
with persistence parameter \(\rho_a\) and target value \(\bar{\nu}>1\).
\(\varepsilon_{a,t}\) and \(\varepsilon_{\nu,t}\) are the exogenous variables.

\paragraph{Calibration}
Use the following values for the model parameters:
\begin{align*}
\beta = 0.96,
\sigma_c = 2,
\sigma_h = 2,
\delta = 0.1,
\chi_h = 1.6, 
\alpha = 0.35,
\rho_a = 0.9,
\bar{\nu} = 5,
\omega = 0.975
\end{align*}

\paragraph{Hints}

\begin{itemize}
\item
A version of the model consists of 11 model equations which are given in
{\eqref{eq:RBCMonopIrrInv.CapitalAccumulation}},
{\eqref{eq:RBCMonopIrrInv.LoM.TFP}},
{\eqref{eq:RBCMonopIrrInv.LoM.Elast}},
{\eqref{eq:RBCMonopIrrInv.LaborSupply}},
{\eqref{eq:RBCMonopIrrInv.EulerCapital}},
{\eqref{eq:RBCMonopIrrInv.KuhnTuckerInvestment}},
{\eqref{eq:RBCMonopIrrInv.RealMarginalCosts}},
{\eqref{eq:RBCMonopIrrInv.MarginalCostsAggregated}},
{\eqref{eq:RBCMonopIrrInv.IntermediateFirms.CapitalLaborRatioAggregated}},
{\eqref{eq:RBCMonopIrrInv.AggregateDemand}}, and
{\eqref{eq:RBCMonopIrrInv.AggregateSupply}}.

\item
Whenever creating plots, use a \emph{reasonable} time horizon
and scale to plot \(3 \times 3=9\) model variables (of your choice).

\item
In Dynare you can write the greater operand \(>\) and the less-or-equal operand \(<=\) just as you do in MATLAB:\\
\texttt{(i > OMEGA*steady{\_}state{(i)})} or \texttt{(i <= OMEGA*steady{\_}state{(i)})}.
\end{itemize}

\bigskip

\begin{center} \Large \textbf{Exercises} \end{center}

\begin{enumerate}

\item
Explain the economic intuition behind including irreversible investments in a DSGE model.

\item
Derive the intratemporal and intertemporal optimality conditions:
\begin{align}
w_t &= \chi_h h_t^{\sigma_h} c_t^{\sigma_c} \label{eq:RBCMonopIrrInv.LaborSupply}
\\
c_t^{-\sigma_c} - \mu_t &= \beta E_t \left[ c_{t+1}^{-\sigma_c} \left( r^k_{t+1} + 1-\delta \right) - (1-\delta) \mu_{t+1}\right]
\label{eq:RBCMonopIrrInv.EulerCapital}
\end{align}
where \(\beta^s \mu_{t+s}\) is the Lagrange multiplier on the occasionally binding constraint~\eqref{eq:RBCMonopIrrInv.IrrInvest} in period \(t+s\).
Note that the complementary slackness condition
\begin{align*}
\mu_t (i_t - \omega \cdot i) &= 0
\end{align*}
can be equivalently written as
\begin{align}
(i_t > \omega \cdot i) \cdot (\mu_t) &+ (i_t \leq \omega \cdot i) \cdot (i_t - \omega \cdot i) = 0  \label{eq:RBCMonopIrrInv.KuhnTuckerInvestment}
\end{align}
Interpret these equations.

\item
Show that profit maximization in the final goods sector implies:
\begin{align}
y_t(f) &= {\left(\frac{P_t(f)}{P_t}\right)}^{-\nu_t} y_t \label{eq:RBCMonopIrrInv.Firms.Demand}
\\
P_t y_t &= \int_{0}^{1} P_t(f) y_t(f) df \label{eq:RBCMonopIrrInv.Firms.ZeroProfit}
\end{align}
Interpret the equation.
What does this imply for real profits \(div_t^{Fin}\) in the final goods sector?

\item
Show that profit maximization in the intermediate good sector implies:
\begin{align}
r^k_t &= mc_t(f) \alpha a_t {\left( \frac{n_t(f)}{k_{t-1}(f)}\right)}^{1-\alpha}
\label{eq:RBCMonopIrrInv.IntermediateFirms.CapitalDemand}
\\
w_t &= mc_t(f) (1-\alpha) a_t {\left(\frac{n_t(f)}{k_{t-1}(f)}\right)}^{-\alpha}
\label{eq:RBCMonopIrrInv.IntermediateFirms.LaborDemand}
\\
P_t(f) &= \frac{\nu_t}{\nu_t - 1} P_t mc_t(f) \label{eq:RBCMonopIrrInv.IntermediateFirms.Price}
\end{align}
where \(mc_t(f)\) is the Lagrange multiplier corresponding to constraint~\eqref{eq:RBCMonopIrrInv.IntermediateFirms.ProductionFunction}.
Interpret the equations.

\item
Show that all intermediate firms choose the same capital to labor ratio in production and marginal costs are the same across firms:
\begin{align}
\left(\frac{k_{t-1}(f)}{n_t(f)}\right) &= \left(\frac{w_t}{1-\alpha}\right) \left(\frac{\alpha}{r^k_t}\right) \label{eq:RBCMonopIrrInv.IntermediateFirms.CapitalLaborRatio}
\\
mc_t \equiv mc_t(f) &= \frac{1}{a_t} {\left(\frac{w_t}{1-\alpha}\right)}^{1-\alpha} {\left(\frac{r^k_t}{\alpha}\right)}^{\alpha} \label{eq:RBCMonopIrrInv.RealMarginalCosts}
\end{align}

\item
Show that aggregation and market clearing implies:
\begin{align}
mc_t = \frac{\nu_t-1}{\nu_t} \label{eq:RBCMonopIrrInv.MarginalCostsAggregated}
\\
\left(\frac{k_{t-1}}{h_t}\right) &= \left(\frac{w_t}{1-\alpha}\right) \left(\frac{\alpha}{r^k_t}\right) \label{eq:RBCMonopIrrInv.IntermediateFirms.CapitalLaborRatioAggregated}
\\
\int_{0}^{1} div_t(f) df &= y_t - w_t h_t - r^k_t k_{t-1}\label{eq:RBCMonopIrrInv.IntermediateFirms.AggregateProfits}
\\
y_t &= c_t + i_t \label{eq:RBCMonopIrrInv.AggregateDemand}
\\
y_t &= a_t k_{t-1}^\alpha h_t^{1-\alpha} \label{eq:RBCMonopIrrInv.AggregateSupply}
\end{align}
Interpret the equations.

\item
Derive with pen and paper the steady-state of the model,
  in the sense that there is a set of values for the endogenous variables that in equilibrium remain constant over time.

\item Write a Dynare mod file that computes the steady-state of this model using an \texttt{initval} block.

\item
Use a perfect foresight simulation for computing the transition path
  for a permanent increase in the elasticity \(\nu_t\) from \(\bar{\nu}\) to \((\bar{\nu}+3)\).
Interpret the economic mechanisms.

\item
Use a perfect foresight simulation for computing the impulse response function of the endogenous variables
  to a \emph{negative} TFP shock on impact of size 0.04.
Interpret the economic mechanisms.

\end{enumerate}

\newpage

\section{Simulating a war shock in a New Keynesian model}
Read the paper by Auray and Eyquem (2019, JEDC): \emph{Episodes of war and peace in an estimated open economy model}.
Particularly, focus on section 3 (The model) and section 6.3 (The macroeconomic consequences of a simulated war episode).

\paragraph{Hints}

\begin{itemize}
\item
A mod file with the model equations and a calibration for France is given in Appendix~\ref{app:AurayEyquem2019.modfile}.

\item
Appendix~\ref{app:AurayEyquem2019.plotfile} contains a helper function for creating plots similar to Figures 3 and 4.

\item
The war shock is in the 9th column of \texttt{oo\_.exo\_simul}.

\item
Set the \texttt{periods} option to 500.
\end{itemize}

\paragraph{Exercises}

\begin{enumerate}

\item
Explain the modelling approach for adding a war into a New Keynesian model.
List all the channels a war affects the economy.
Do you find these convincing or not?

\item
Replicate Figures 3 and 4 of the paper using a deterministic simulation under perfect foresight \textbf{for France only}.
To this end, use Dynare's \texttt{perfect\_foresight\_setup} and \texttt{perfect\_foresight\_solver} commands
  to simulate an unanticipated war which, as soon as it occurs, is known to last for 5 years.

\item
The authors claim that they are doing a \emph{MIT} shock.
Research and explain the concept of a MIT shock in the context of deterministic simulations in DSGE models.

\item
Dynare 6.0 introduced new commands \texttt{perfect\_foresight\_with\_expectation\_errors\_setup} and \texttt{perfect\_foresight\_with\_expectation\_errors\_solver}.
Consult the Dynare manual for their syntax and use.
Then redo the simulation (for France only) with a sequence of 5 consecutive war shocks, ensuring that the agents are continually surprised by both the onset and continuation of war.
Create the plots using the helper function.
    
\end{enumerate}

\newpage

\appendix

\section{Auray and Eyquem (2019) mod file\label{app:AurayEyquem2019.modfile}}
\lstinputlisting[style=Matlab-editor,basicstyle=\mlttfamily\scriptsize,title=\lstname]{../progs/replications/Auray_Eyquem_2019/Auray_Eyquem_2019.mod}

\section{Auray and Eyquem (2019) helper function for plots\label{app:AurayEyquem2019.plotfile}}
\lstinputlisting[style=Matlab-editor,basicstyle=\mlttfamily\scriptsize,title=\lstname]{../progs/replications/Auray_Eyquem_2019/Auray_Eyquem_2019_plots.m}

\end{document}
