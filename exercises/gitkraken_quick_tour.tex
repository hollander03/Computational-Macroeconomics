\section[Quick Tour: git with GitKraken]{Quick Tour: git with GitKraken (OPTIONAL)\label{ex:QuickTourGitWithGitKraken}}

\begin{enumerate}

\item
What is \emph{git}? What is \emph{GitKraken}? What is \emph{GitHub}?

\item
How does a typical workflow in Computational Macroeconomics look like?
How can a version control system like git support this workflow?

\item
If you haven't already, sign up for a free account on GitHub: \url{https://github.com/signup}

\item
If you haven't already, sign up for the GitHub Student Developer Pack \url{https://education.github.com/pack}.
If you add your university email address, the decision process is very fast (usually within a day).

\item
Install GitKraken for your operating system \url{https://www.gitkraken.com/download} and connect it with GitHub.
Go to preferences \(\rightarrow \) Integrations \(\rightarrow \) GitHub and click \emph{Generate SSH key and add to GitHub}.
Checkout the other preferences.
For instance, if you are on Windows make sure that under \emph{Editor} the \emph{End of Line Character} is set to \emph{LF}.\footnote{%
  If you are on Windows and working with people who are not (or vice-versa),
    you'll probably run into line-ending issues at some point.
  This is because Windows uses both a carriage-return character and a linefeed character for newlines in its files,
    whereas macOS and Linux systems use only the linefeed character.
  This is a subtle but incredibly annoying fact of cross-platform work;
    many editors on Windows silently replace existing LF-style line endings with CRLF,
    or insert both line-ending characters when the user hits the enter key.
  Therefore, it is advised to tell git to convert CRLF to LF on commit but not the other way around.
}

\item
Create a local repository on your computer called \emph{Comp-Macro} by selecting \texttt{Start a local repo} in GitKraken.

\item
With your favorite text editor open the \texttt{README.md} file inside the folder and add the following:
\begin{lstlisting}[basicstyle=\scriptsize\mlttfamily]
# Repository for Computational Macroeconomics
In this repository I will practice solving DSGE models.
The folder contains examples and codes developed in the lecture.
I also don't understand git and find this really cumbersome!
Dropbox, iCloud, Nextcloud, and OneDrive is so much better!
\end{lstlisting}
Save the file and have a look into GitKraken what happened in your repository.

\item
Explain the \emph{git model} of staging, committing and pushing.

\item
Stage and commit all the changes you made so far to the repository.

\item
What is a \emph{good commit}?

\item
\emph{Soft Reset} your current status (so-called HEAD) to the initial commit.
Re-commit everything except the last line in the README.md file without actually changing the file in your text editor.
Put the remaining changes (i.e.\ the last line into the stash, which is a place for work-in-progress).

\item
Push your local repository to GitHub and visit https://github.com/[USERNAME]/Comp-Macro to see what happened.

\item
What are \emph{git branches}?
Create a branch called \emph{latex-exam-template} and checkout the branch.

\item
While on your \emph{latex-exam-template} branch, use your favorite text editor
  to create the three files \texttt{templateExamSolution.tex}, \texttt{templateExamBiblio.bib}, and \texttt{templateMatlabExample.m} given in the Appendix.
Typically, when copy and pasting from a pdf you will run into characters (like minus signs and empty space) not correctly pasted into your text editor,
  so you will need to check everything manually.
Once you are happy, commit only three files using the message \enquote{Created latex template for exam solutions}.

\item
Compile the Latex files using your favorite Latex GUI or via the command line:
\begin{lstlisting}[language=tex,frame=single,basicstyle=\scriptsize\mlttfamily]
pdflatex templateExamSolution
biber templateExamSolution
pdflatex templateExamSolution
pdflatex templateExamSolution
\end{lstlisting}
Open the pdf to make sure the file compiled correctly.

\item
Like in the previous exercise we often have some auxiliary files created by a program which we do not care about.
We can tell \emph{git} to ignore individual files or patterns of files.
In GitKraken, simply right-click on the file you want to ignore and select how you want to ignore it.
Do so for the just created auxiliary Latex files, but also for the generated pdf (as it is binary).
Have a look into the newly created \texttt{.gitignore} file and commit it.\footnote{%
  GitHub also has a collection of useful gitignore files for various programming languages at \url{https://github.com/github/gitignore}.
}

\item
Create a Pull Request to GitHub of your develop branch to your main branch. Go to GitHub and accept this.
See what happens in the repository.

\item
Switch back to your main branch and pull the merged changes.
Then click Pop to add the changes you had in your stash back on top of the current branch.
Either decide to discard the last line in the README.md file or commit it and schedule a meeting with me to discuss your issues.

\item
Once you got accepted to the GitHub Student Developer Pack, activate GitKraken to get the free PRO license.

\end{enumerate}


\paragraph{Readings}
\begin{itemize}
\item \url{https://www.gitkraken.com/learn/git}: highly recommended (particularly the videos)
\item \textcite{Chacon_2014_ProGit}: extensive book on git
\end{itemize}

\paragraph{Useful urls}
\begin{itemize}	
\item \url{https://www.sas.upenn.edu/~jesusfv/Chapter_HPC_5_Git.pdf}
\item \url{https://github.com/jaredgars/LEAP}
\item \url{https://luispfonseca.com/files/slides_git.pdf}
\item \url{https://www.frankpinter.com/notes/git-for-economists-presentation.pdf}
\item \url{https://github.com/fditraglia/git-for-economists}
\item \url{https://matteosostero.com/files/slides_git.pdf}
\end{itemize}

\begin{solution}\textbf{Solution to \nameref{ex:QuickTourGitWithGitKraken}}
\ifDisplaySolutions%
\input{exercises/gitkraken_quick_tour_solution.tex}
\fi
\newpage
\end{solution}