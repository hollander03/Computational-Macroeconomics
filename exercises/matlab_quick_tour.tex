\section[Quick Tour: MATLAB]{Quick Tour: MATLAB\label{ex:QuickTourMATLAB}}
Install the most recent version of MATLAB with the following Toolboxes:
  Econometrics Toolbox,
  Global Optimization Toolbox,
  Optimization Toolbox,
  Parallel Computing Toolbox,
  Statistics and Machine Learning Toolbox,
  Symbolic Math Toolbox.
Open a new script and do the following:\footnote{%
  If you don't have any previous programming experience,
  I would highly recommend to go through
  Appendix A of \textcite{Brandimarte_2006_NumericalMethodsFinance},
  Appendix B of \textcite{Miranda.Fackler_2002_AppliedComputationalEconomics},
  and \textcite{Pfeifer_2017_MATLABHandout}.
}

\begin{enumerate}

\item
Define the column vectors
\begin{align*}
x &= \left(-1,0,1,4,9,2,1,4.5,1.1,-0.9\right)' && y =\left(1,1,2,2,3,3,4,4,5,\text{nan}\right)'
\end{align*}

\item
Check if both vectors have the same length using either \texttt{length{()}} or \texttt{size{()}}.

\item
Perform the following logical operations:
\begin{align*}
x&<y && x<0 && x+3\geq0 && y<0
\end{align*}

\item
Check if all elements in x satisfy both \(x + 3 \geq 0\) and \(y > 0\).

\item
Check if all elements in x satisfy either \(x + 3 \geq 0\) or \(y > 0\).

\item
Check if at least one element of \(y\) is greater than \(0\).

\item
Compute \(x+y\), \(xy\), \(xy'\), \(x'y\), \(y/x\), and \(x/y\).

\item
Compute the element-wise product and division of \(x\) and \(y\).

\item
Compute \(\ln(x)\) and \(e^x\).

\item
Use \texttt{any} to check if the vector \(x\) contains elements satisfying \(\sqrt{x} \geq 2\).

\item
Compute \(a = \sum_{i=1}^{10} x_{i}\) and \(b = \sum_{i=1}^{10} y_{i}^{2}\).
Omit the nan in \(y\) when computing the sum.

\item
Compute \(\sum_{i=1}^{10} x_{i} y_{i}^{2}\).
Omit the nan in \(y\) when computing the sum.

\item
Count the number of elements of \(x>0\).

\item
Predict what the following commands will return:
\begin{verbatim}
x.^y  x.^(1/y)  log(exp(y))  y*[-1,1]  x+[-1,0,1]  sum(y*[-1,1],1,'omitnan')
\end{verbatim}

\item
Define the matrix \(X=\begin{pmatrix} 1 & 4 & 7 \\ 2 & 5 & 8 \\ 3 & 6 & 9 \end{pmatrix}\).
Print the transpose, dimensions and determinant of \(X\).

\item
Compute the trace of \(X\) (i.e.\ the sum of its diagonal elements).

\item
Change the diagonal elements of \(X\) to [7,8,9].

\item
Compute the eigenvalues of (the new) \(X\).
Display a message if \(X\) is positive or negative definite.

\item
Invert \(X\) and compute the eigenvalues of \(X^{-1}\).

\item
Define the column vector \(a = (1,3,2)'\) and compute \texttt{a'*X}, \texttt{a'.*X}, and \texttt{X*a}.

\item
Compute the quadratic form \(a^{\prime }Xa\).

\item
Define the matrices
\(I = \left[\begin{array}{lll} 1 & 0 & 0 \\ 0 & 1 & 0 \\ 0 & 0 & 1 \end{array}\right]\),
\(Y = \left[\begin{array}{lll} 1 & 4 & 7 \\ 2 & 5 & 8 \\ 3 & 6 & 9 \end{array}
            \begin{array}{lll} 1 & 0 & 0 \\ 0 & 1 & 0 \\ 0 & 0 & 1 \end{array}\right]\)
and
\(Z = \left[\begin{array}{lll} 1 & 4 & 7 \\ 2 & 5 & 8 \\ 3 & 6 & 9 \\ 1 & 0 & 0 \\ 0 & 1 & 0 \\ 0 & 0 & 1 \end{array}\right]\).

\item
Generate the vectors
\begin{align*}
x_{1} &= \left(  1, 2,  3, \ldots,  9          \right) ,&&
x_{2}  = \left(  0, 1,  0,      1,  0, 1, 0, 1 \right) ,&&
x_{3}  = \left(  1, 1,  1,      1,  1, 1, 1, 1 \right) \\
x_{4} &= \left( -1, 1, -1,      1, -1, 1       \right) ,&&
x_{5}  = \left( 1980, 1985, 1990, \ldots, 2010 \right) ,&&
x_{6}  = \left( 0, 0.01, 0.02, \ldots, 0.99, 1 \right)
\end{align*}
using sequence operator \texttt{:} or \texttt{repmat}.

\item
Generate a grid of \(n=500\) equidistant points on the interval \([-\pi,\pi]\) using \texttt{linspace}.

\item
Compare \texttt{1:10+1}, \texttt{(1:10)+1} and {\texttt{1}:\texttt{(10+1)}}.

\item
Define the following column vectors
\begin{align*}
x &= \begin{pmatrix} 1 & 1.1 & 9 & 8 & 1 & 4 & 4 & 1 \end{pmatrix}
,\quad
y = \begin{pmatrix} 1 & 2 & 3 & 4 & 4 & 3 & 2 & \text{NaN} \end{pmatrix}'
\\
z &= \begin{pmatrix} true & true & false & false & true & false & false & false	\end{pmatrix}'
\end{align*}

\item
Predict what the following commands will return (and then check if you are right):
\\
\texttt{x{(2:5)}}, \texttt{x{(4:end-2)}}, \texttt{x{([1 5 8])}}, \texttt{x{(repmat{(1:3,1,4)})}},
\\
\texttt{y{(z)}}, \texttt{y{(\textasciitilde{z})}}, \texttt{y{(x>2)}}, \texttt{y{(x==1)}},
\\
\texttt{x{(\textasciitilde{isnan{(y)}})}}, \texttt{y{(\textasciitilde{isnan{(y)}})}}

\item
Indexing is not only used to read certain elements of a vector but also to change them.
Execute \texttt{x2 = x} to make a copy of \texttt{x}.
Change all elements of \texttt{x2} that have the value \(4\) to the value \(-4\).
Print \texttt{x2}.

\item
Change all elements of \texttt{x2} that have the value 1 to a missing value (\texttt{nan}).
Print \texttt{x2}.

\item
Execute \texttt{x2(z) = []}. Print \texttt{x2}.

\item
Define the matrix
\(M = \begin{pmatrix}
1 & 5 &  9 & 12 & 8 & 4\\
2 & 6 & 10 & 11 & 7 & 3\\
3 & 7 & 11 & 10 & 6 & 2\\
4 & 8 & 12 &  9 & 5 & 1
\end{pmatrix}\)
using the \({:}\) operator and the \texttt{reshape} command.

\item
Predict what the following commands will return (and then check if you are right):
\\
\texttt{M{(1,3)}}, \texttt{M{(:,5)}}, \texttt{M{(2,:)}}, \texttt{M{(2:3,3:4)}}, \texttt{M{(2:4,4)}},
\texttt{M{(M>5)}}, \texttt{M{(:,M{(1,:)}<=5)}}, \texttt{M{(M{(:,2)}>6,:)}}, \texttt{M{(M{(:,2)}>6,4:6)}}

\item
Print all rows of \(M\) where column 5 is at least three times larger than column 6.

\item
Count the number of elements of \(M\) that are larger than 7.

\item
Count the number of elements of \(M\) in row 2 that are smaller than their neighbors in row 1.

\item
Count the number of elements of \(M\) that are larger than their left neighbor.

\end{enumerate}


\paragraph{Readings}
\begin{itemize}
\item \textcite[Appendix A]{Brandimarte_2006_NumericalMethodsFinance}.
\item \textcite[Appendix B]{Miranda.Fackler_2002_AppliedComputationalEconomics}.
\item \textcite{Pfeifer_2017_MATLABHandout}
\end{itemize}

\begin{solution}\textbf{Solution to \nameref{ex:QuickTourMATLAB}}
\ifDisplaySolutions%
\input{exercises/matlab_quick_tour_solution.tex}
\fi
\newpage
\end{solution}