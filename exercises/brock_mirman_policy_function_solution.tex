\begin{enumerate}

\item Due to our assumptions , we will not have corner solutions and can neglect the non-negativity constraints.
Due to the transversality condition and the concave optimization problem, we only need to focus on the first-order conditions. 
The Lagrangian for the household problem is
\begin{align*}
L = E_t\sum_{j=0}^{\infty}\beta^j\left\{\log(c_{t+j}) + \lambda_{t+j} \left(a_{t+j}k_{t+j-1}^\alpha - c_{t+j} - k_{t+j}\right)\right\}
\end{align*}
Note that the problem is not to choose $\{c_t,k_{t}\}_{t=0}^\infty$ all at once in an open-loop policy,
  but to choose these variables sequentially given the information at time $t$ in a closed-loop policy.

The first-order condition w.r.t. $c_t$ is given by
\begin{align*}
\frac{\partial L}{\partial c_{t}} &= E_t \left(c_t^{-1}-\lambda_{t}\right) = 0
\\
\Leftrightarrow \lambda_{t} &= c_{t}^{-1} & (I)
\end{align*}
The first-order condition w.r.t. $k_{t}$ is given by
\begin{align*}
\frac{\partial L}{\partial k_{t}} &= E_t (-\lambda_{t}) + E_t \beta \left(\lambda_{t+1}\alpha a_{t+1} k_{t}^{\alpha-1}\right) = 0
\\
\Leftrightarrow \lambda_{t} &= \alpha\beta E_t \lambda_{t+1} a_{t+1} k_{t}^{\alpha-1} & (II)
\end{align*}
			
(I) and (II) yields
\begin{align*}
c_t^{-1} = \alpha \beta E_t c_{t+1}^{-1} a_{t+1} k_{t}^{\alpha-1}
\end{align*}
	
\item First, consider the steady value of technology: 
\begin{align*}
\log\bar{a}=\rho \log\bar{a} + 0 \Leftrightarrow \log\bar{a} = 0 \Leftrightarrow \bar{a} = 1
\end{align*}
The Euler equation in steady-state becomes:
\begin{align*}
\bar{k} = (\alpha \beta \bar{a})^{\frac{1}{1-\alpha}}
\end{align*}
	
\item Inserting the guessed policy function for $c_t$ inside the the capital accumulation equation yields:
\begin{align*}
k_{t} = a_{t} k_{t-1}^\alpha - g^c a_t k_{t-1}^\alpha = (1-g^c) a_t k_{t-1}^\alpha
\end{align*}
Therefore, $g^k=(1-g^c)$; in other words, once we derive $g^c$, we get $g^k$.
			
Inserting the guessed policy function for $c_t$ inside the Euler equation yields
\begin{align*}
\frac{1}{c_t} = \alpha \beta E_t \frac{1}{c_{t+1}} a_{t+1} k_{t}^{\alpha-1}
\\
\frac{1}{g^c a_t k_{t-1}^\alpha} = \alpha \beta E_t \frac{1}{g^c a_{t+1} k_{t}^\alpha} a_{t+1} k_{t}^{\alpha-1}
\\
a_t k_{t-1}^\alpha = \frac{1}{\alpha \beta} E_t k_{t}
\end{align*}
Inserting the decision rule for capital:
\begin{align*}
a_t k_{t-1}^\alpha = \frac{1}{\alpha \beta} (1-g^c)a_t k_{t-1}^\alpha
\\
\Leftrightarrow g^c = (1-\alpha \beta)
\end{align*}
Thus the policy functions for $c_t$ and $k_t$ are
\begin{align*}
c_t &= (1-\alpha\beta) a_t k_{t-1}^\alpha
\\
k_{t} &= \alpha \beta a_t k_{t-1}^\alpha
\end{align*}
Now inserting $a_t = a_{t-1}^{\rho} e^{\varepsilon_t}$ yields:
\begin{align*}
a_t &= a_{t-1}^{\rho} e^{\varepsilon_t}
\\
c_t &= (1-\alpha\beta) a_{t-1}^\rho k_{t-1}^\alpha e^{\varepsilon_t}
\\
k_t &= \alpha\beta a_{t-1}^\rho k_{t-1}^\alpha e^{\varepsilon_t}
\end{align*}
In summary we have found analytically the policy functions.
This will not be possible for other DSGE models and we have to rely on numerical methods to approximate it.

\item The following codes illustrate the exercise both with Dynare as well as in MATLAB,
  and compares the results to show that they are (numerically equivalent):

\lstinputlisting[style=Matlab-editor,basicstyle=\mlttfamily\scriptsize,title=\lstname]{progs/matlab/Brock_Mirman_compare.m}

Here is the Dynare mod file:
\lstinputlisting[style=Matlab-editor,basicstyle=\mlttfamily\scriptsize,title=\lstname]{progs/dynare/Brock_Mirman.mod}

Here is the manual preprocessing of the model in MATLAB:
\lstinputlisting[style=Matlab-editor,basicstyle=\mlttfamily\scriptsize,title=\lstname]{progs/matlab/Brock_Mirman_preprocessing.m}

Here is the steady-state script in MATLAB:
\lstinputlisting[style=Matlab-editor,basicstyle=\mlttfamily\scriptsize,title=\lstname]{progs/matlab/Brock_Mirman_ss.m}

\paragraph{Interpretation} A first-order perturbation solution is already very accurate for this model if the model dynamics stay close to the steady-state.
In other words, only for large shocks (try increasing $\sigma$) we see that the approximated and true trajectories differ in size, but not in direction.

\end{enumerate}